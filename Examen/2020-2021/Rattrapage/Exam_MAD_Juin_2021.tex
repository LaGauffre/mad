\documentclass[11pt, addpoints, answers]{exam}

\usepackage[utf8]{inputenc}
\usepackage[T1]{fontenc}
\usepackage[margin  = 1in]{geometry}
\usepackage{amsmath, amscd, amssymb, amsthm, verbatim}
\usepackage{mathabx}
\usepackage{setspace}
\usepackage{float}
\usepackage{color}
\usepackage{graphicx}
\usepackage[colorlinks=true]{hyperref}
\usepackage{tikz}

\usetikzlibrary{shapes,arrows}
%%%<
\usepackage{verbatim}
%%%>
\usetikzlibrary{automata,arrows,positioning,calc}

\usetikzlibrary{trees}

\shadedsolutions
\definecolor{SolutionColor}{RGB}{214,240,234}

\newcommand{\bbC}{{\mathbb C}}
\newcommand{\R}{\mathbb{R}}            % real numbers
\newcommand{\bbR}{{\mathbb R}}
\newcommand{\Z}{\mathbb{Z}}            % integers
\newcommand{\bbZ}{{\mathbb Z}}
\newcommand{\bx}{\mathbf x}            % boldface x
\newcommand{\by}{\mathbf y}            % boldface y
\newcommand{\bz}{\mathbf z}            % boldface z
\newcommand{\bn}{\mathbf n}            % boldface n
\newcommand{\br}{\mathbf r}            % boldface r
\newcommand{\bc}{\mathbf c}            % boldface c
\newcommand{\be}{\mathbf e}            % boldface e
\newcommand{\bE}{\mathbb E}            % blackboard E
\newcommand{\bP}{\mathbb P}            % blackboard P

\newcommand{\ve}{\varepsilon}          % varepsilon
\newcommand{\avg}[1]{\left< #1 \right>} % for average
%\renewcommand{\vec}[1]{\mathbf{#1}} % bold vectors
\newcommand{\grad}{\nabla }
\newcommand{\lb}{\langle }
\newcommand{\rb}{\rangle }

\def\Bin{\operatorname{Bin}}
\def\Var{\operatorname{Var}}
\def\Geom{\operatorname{Geom}}
\def\Pois{\operatorname{Pois}}
\def\Exp{\operatorname{Exp}}
\newcommand{\Ber}{\operatorname{Ber}}
\def\Unif{\operatorname{Unif}}
\def\No{\operatorname{N}}
\newcommand{\E}{\mathbb E}            % blackboard E
\def\th{\theta }            % theta shortcut
\def\V{\operatorname{Var}}
\def\Var{\operatorname{Var}}
\def\Cov{\operatorname{Cov}}
\def\Corr{\operatorname{Corr}}
\newcommand{\epsi}{\varepsilon}            % epsilon shortcut

\providecommand{\norm}[1]{\left\lVert#1\right\rVert} %norm
\providecommand{\abs}[1]{\left \lvert#1\right \rvert} %absolute value

\DeclareMathOperator{\lcm}{lcm}
\newcommand{\ds}{\displaystyle}	% displaystyle shortcut

\def\semester{2020-2021}
\def\course{Modèles Aléatoires Discrets}
\def\title{\MakeUppercase{Examen de deuxième session}}
\def\name{Pierre-O Goffard}


\setlength\parindent{0pt}

\cellwidth{.35in} %sets the minimum width of the blank cells to length
\gradetablestretch{2.5}

%\bracketedpoints
%\pointsinmargin
%\pointsinrightmargin

\begin{document}


\runningheader{\course  \vspace*{.25in}}{}{\title \vspace*{.25in}}
%\runningheadrule
\runningfooter{}{Page \thepage\ of \numpages}{}

% \firstpageheader{Name:\enspace\hbox to 2.5in{\hrulefill}\\  \vspace*{2em} Section: (circle one) TR: 3-3:50 \textbar\, TR: 5-5:50 \textbar\,  TR: 6-6:50(Xu) \textbar\,  TR: 6-6:50 }{}{Perm \#: \enspace\hbox to 1.5in{\hrulefill}\\ \vspace*{2em} Score:\enspace\hbox to .6in{\hrulefill} $/$\numpoints}
\extraheadheight{.25in}

\hrulefill

\vspace*{1em}

% Heading
{\center \textsc{\Large\title}\\
	\vspace*{1em}
	\course -- \semester\\
	Pierre-O Goffard\\
}
\vspace*{1em}

\hrulefill

\vspace*{2em}

\noindent {\bf\em Instructions:} On éteint et on range son téléphone.
\begin{itemize}
	\item La calculatrice et les appareils éléctroniques ne sont pas autorisés.
	\item Vous devez justifier vos réponses de manière claire et concise.
	\item Vous devez écrire de la manière la plus lisible possible. Souligner ou encadrer votre réponse finale.

\end{itemize}

\begin{center}
	\gradetable[h]
\end{center}

\smallskip

\begin{questions}
\question[2] Rappeler ce qu'est une mesure réversible pour une chaine de Markov homogène $(X_n)_{n\geq0}$ d'espace d'état $E$ et de probabilités de transition $Q(x,y),\text{ pour }x,y\in E$. Montrer qu'une loi reversible est aussi une mesure invariante.
\begin{solution}
Voir le cours
\end{solution}
\question Soit $(X_n)_{n\geq0}$ une chaine de markov homogène sur un espace d'état $E= \{1,2,3\}$ et de matrice des tranition
\[
Q =\left(\begin{array}{ccc}
0&1/2&1/2\\
1/2&0&1/2\\
1&0&0
\end{array}\right)
\]
\begin{parts}
\part[1] Identifier les classes de communications, sont-elles ouvertes ou fermées?
\begin{solution}
Une seule classe de communication $\{1,2,3\}$ fermée
\end{solution}
\part[2] Après avoir justifié son existence et son unicité, donner la loi de probabilité invariante de $(X_n)_{n\geq0}$.
\begin{solution}
L'espace d'état de $(X_n)_{n\geq0}$ est fini, il existe donc une loi de probabilité invariante. Comme la chaine est irréductible alors cette loi de probabilité est unique. On résout le système
$$
\begin{cases}
\pi Q = \pi\\
\sum_{x\in E} \pi(x) = 1
\end{cases}
$$
et on obtient $\pi = \left(\begin{array}{ccc}4/9&2/9&3/9\end{array}\right)$
\end{solution}
\part[1] Donner la période de chaque état $x\in E$. Que vaut la limite $\underset{n\rightarrow+\infty}{\lim}Q^n(x,y)$?
\begin{solution}
On a 
$$1\rightarrow 3\rightarrow 1\text{ et }1\rightarrow 2\rightarrow 3\rightarrow 1.$$
On en déduit qu ela période de l'état $1$ est donnée par 
$$
d(1) = \text{pgcd}\{2,3,\ldots,\} = 1.
$$
Comme la chaine est irréductible alors tous les états ont la même période 
$$d(2)=d(3) = d(1) = 1.$$
Comme la chaine est apériodique, irréductible et récurrente alors 
$$
\underset{n\rightarrow+\infty}{\lim}Q^n(x,y) = \pi(y)
$$ 
\end{solution}
\part[2] Soit $T_3  = \inf\{n\geq0\text{ ; }X_n = 3\}$, calculer 
$$
\mathbb{E}_x(T_3) = \mathbb{E}(T_3|X_0 = x),\text{ pour tout }x\in E. 
$$
\begin{solution}
On effectue une analyse à un pas, on résout 
$$
\begin{cases}
\E_1(T_3) = 1+\frac{1}{2}\E_2(T_3)\frac{1}{2}\E_3(T_3)\\
\E_2(T_3) = 1+\frac{1}{2}\E_1(T_3)\frac{1}{2}\E_3(T_3)\\
\E_3(T_3) = 0
\end{cases}
$$
Il vient $E_1(T_3)=\E_2(T_3) = 2$.
\end{solution}
\end{parts}
\question Un bûcheron possède un stock de bois dont le nombre d'unité (bûche de bois) est donné par $X_n$ à l'instant $n\geq0$. Il consomme une unité de bois par jour. Lorsque sa réserve de bois tombe à $0$ il part couper du bois le jour suivant et ramène à la maison $Y$ unités de bois où $Y$ est une variable aléatoire à valeur entière de loi de probabilités
\[
\mathbb{P}(Y = y) = p(y)>0,\text{ }y\geq1,
\]
avec $\sum_{y\geq1}p(y)=1$. A noter que le jour pendant lequel il part couper du bois, il consomme aussi une unité de bois
\begin{parts}
\part[2] Donner l'espace d'état et les probabilités de transition de la chaine de Markov homogène $(X_n)_{n\geq0}$.
\begin{solution}
$E=\mathbb{N}$ et 
$$
Q(x,y) = \begin{cases}
1&\text{si }x>0\text{ et } y = x-1,\\
p(y+1)&\text{si }x=0\text{ and }y\geq1,\\
0&\text{ sinon.}

\end{cases}
$$
\end{solution}
\part[1] La chaine $(X_n)_{n\geq0}$ est elle irréductible?
\begin{solution}
Soit $n = x+1$, on a 
$$
Q^n(x,y) =Q^{x+1}(x,y) = Q^x(x,0)Q(0,y) = p(y+1)>0 
$$
La chaine est bien irréductible.
\end{solution}
\part[1] La chaine $(X_n)_{n\geq0}$ est elle récurrente?
\begin{solution}
On a 
$$
\sum_{n\geq1}Q^{n}(x,x)\geq  \sum_{n\geq x}Q^{n}(x,x)\geq\sum_{n\geq x}p(x+1)=\infty
$$
L'état $x\in E$ est récurrent et puisque la chaine est irréductible alors tous les états sont récurrents.
\end{solution}
\part[1] Montrer que la mesure 
\[
\lambda(x) = \sum_{z \geq x+1}p(z),\text{ }x\geq0
\]
est une mesure invariante de $(X_n)_{n\geq0}$.
\begin{solution}
On a 
$$
\sum_{x\geq0}\lambda(x)Q(x,y) = \lambda(0)p(y+1)+ \lambda(y+1) = \lambda(y)
$$
$\lambda$ est bien une mesure invariante.
\end{solution}
\part[2] Discuter de l'existence et de l'unicité d'une loi de probabilité invariante en fonction des réponses aux questions (b), (c) et d'une condition sur $\mathbb{E}(Y)$. Si elle existe, donner cette mesure de probabilité invariante.  
\begin{solution}
Comme la chaine est iréductible et récurrente alors toutes les lois sont proportionnelles. Si la mesure $\lambda$ est finie alors on peut la normalisé. On note que $\sum_{x\in E} \lambda(x) = \mathbb{E}(Y)$. Si $\E(Y)<\infty$ alors la probabilité invariante est unique et donnée par 
$$
\pi(x) = \lambda(x)/\E(Y),\text{ }x\in E,
$$
sinon la chaine est récurrente nulle et il n'y a pas de loi de probabilité invariante.
\end{solution}
\end{parts}
\end{questions}
%-------------------------------TABLE-------------------------------
\newpage
\hrule
\vspace*{.15in}
\begin{center}
  \large\MakeUppercase{Formulaire}
\end{center}
\vspace*{.15in}
\hrule
\vspace*{.25in}

\renewcommand\arraystretch{3.5}
\begin{table}[H]
\begin{center}
\footnotesize
\begin{tabular}{|c|c|c|c|c|c|}

\hline
Nom & abbrev. & Loi & $\E(X)$ & $\Var(X)$ & FGM\\
\hline\hline
Binomial & $\Bin(n,p)$ & $\binom{n}{k}p^k(1-p)^{n-k}$ & $np$ & $np(1-p)$ & $[(1-p)+pe^t]^n$\\
\hline
Poisson & $\Pois(\lambda)$ & $e^{-\lambda}\dfrac{\lambda^k}{k!}$ & $\lambda$ & $\lambda$ &$ \exp(\lambda(e^t-1))$\\
\hline
Geometric & $\Geom(p)$ & $(1-p)^{k-1}p$ & $\dfrac{1}{p}$ & $\dfrac{1-p}{p^2}$ & $\frac{pe^t}{1-(1-p)e^t}$ pour  $t<-\ln(1-p)$\\
\hline
Uniform & $\Unif(a,b)$ & $\begin{cases} \dfrac{1}{b-a} & a\leq t\leq b\\ 0 & \text{sinon}\end{cases}
$ & $\dfrac{a+b}{2}$ & $\dfrac{(b-a)^2}{12}$ & $\frac{e^{tb}-e^{ta}}{t(b-a)}$\\
\hline
Exponential & $\Exp(\lambda)$ & $\begin{cases} \lambda e^{-\lambda t} & t\geq 0 \\ 0 & t<0\end{cases}$ & $\dfrac{1}{\lambda}$ & $\dfrac{1}{\lambda^2}$ & $\frac{\lambda}{\lambda -t}$ pour $t<\lambda$\\
\hline
Normal & $\No(\mu,\sigma^2)$ & $\left(\dfrac{1}{\sqrt{2\pi\sigma^2}}\right)\operatorname{exp}{\left(\dfrac{-(t-\mu)^2}{2\sigma^2}\right)}$ & $\mu$ & $\sigma^2$ & $e^{\mu t}e^{\sigma^2t^2/2}$\\
\hline
\end{tabular}
\end{center}
\end{table}%

\end{document}