\documentclass[8pt,notheorems]{beamer}
\usetheme{Copenhagen}
\usepackage[utf8]{inputenc}
\usepackage[T1]{fontenc}
\usepackage{beamerthemesplit}
\usepackage{graphicx}
\usepackage{tkz-graph}
\usepackage{color}
\usepackage{listings}

\usepackage{amsmath,amsfonts,amsthm,t1enc}
\usepackage{fourier}
\usepackage{listings}
\usepackage{amsmath}
\usepackage{tikz}
\usetikzlibrary{automata,arrows,positioning,calc}
\setbeamertemplate{footline}{\hfill \insertframenumber/\inserttotalframenumber}
\def \si {\sigma}
\def \la {\lambda}
\def \al {\alpha}
% \def\e*{\end{eqnarray*}}
\def \di{\displaystyle}

\def \E{\mathbb E}
\def \N{\mathbb N}
\def \NZ{\mathbb{N}_0}
\def \I{\mathbb I}
\def \w{\widehat}
\def \P {\mathbb P}
\def \V{\mathbb V}


\newcommand{\CL}{\mathbb{C}}
\newcommand{\RL}{\mathbb{R}}
\newcommand{\nat}{{\mathbb N}}
\newcommand{\Laplace}{\mathscr{L}}
\newcommand{\e}{\mathrm{e}}
\newcommand{\ve}{\bm{\mathrm{e}}} % vector e

\renewcommand{\L}{\mathcal{L}} % e.g. L^2 loss.

\newcommand{\ih}{\mathrm{i}}
\newcommand{\oh}{{\mathrm{o}}}
\newcommand{\Oh}{{\mathcal{O}}}
\newcommand{\Exp}{\mathbb{E}}

\newcommand{\Norm}{\mathcal{N}}
\newcommand{\LN}{\mathcal{LN}}
\newcommand{\SLN}{\mathcal{SLN}}

\renewcommand{\Pr}{\mathbb{P}}
\newcommand{\Ind}{\mathbb I}
\newcommand\bfsigma{\bm{\sigma}}
\newcommand\bfSigma{\bm{\Sigma}}
\newcommand\bfLambda{\bm{\Lambda}}
\newcommand{\stimes}{{\times}}

\setbeamertemplate{theorem}[ams style]
\setbeamertemplate{theorems}[numbered]
%\makeatletter
%\def\th@mystyle{%
%    \normalfont % body font
%    \setbeamercolor{block title example}{bg=orange,fg=white}
%    \setbeamercolor{block body example}{bg=blue!20,fg=black}
%    \def\inserttheoremblockenv{block}
%  }
%\makeatother
%\theoremstyle{mystyle}

\makeatletter
    \ifbeamer@countsect
      \newtheorem{theorem}{\translate{Theorem}}[section]
    \else
      \newtheorem{theorem}{\translate{Theorem}}
    \fi
    \newtheorem{corollary}{\translate{Corollary}}
    \newtheorem{prop}{\translate{Proposition}}
    \newtheorem{lemma}{\translate{Lemma}}
    \newtheorem{problem}{\translate{Problem}}
    \newtheorem{solution}{\translate{Solution}}

    \theoremstyle{definition}
    \newtheorem{definition}{\translate{Definition}}
    \newtheorem{definitions}{\translate{Definitions}}

    \theoremstyle{example}
    \newtheorem{example}{\translate{Example}}
    \newtheorem{examples}{\translate{Examples}}

\makeatletter
\def\th@mystyle{%
    \normalfont % body font
    \setbeamercolor{block title example}{bg=orange,fg=white}
    \setbeamercolor{block body example}{bg=orange!20,fg=black}
    \def\inserttheoremblockenv{exampleblock}
  }
\makeatother
\theoremstyle{mystyle}
\newtheorem{fact}{Fact}





    % Compatibility
    \newtheorem{Beispiel}{Beispiel}
    \newtheorem{Beispiele}{Beispiele}
    \theoremstyle{plain}
    \newtheorem{Loesung}{L\"osung}
    \newtheorem{Satz}{Satz}
    \newtheorem{Folgerung}{Folgerung}
    \newtheorem{Fakt}{Fakt}
    \newenvironment{Beweis}{\begin{proof}[Beweis.]}{\end{proof}}
    \newenvironment{Lemma}{\begin{lemma}}{\end{lemma}}
    \newenvironment{Proof}{\begin{proof}}{\end{proof}}
    \newenvironment{Theorem}{\begin{theorem}}{\end{theorem}}
    \newenvironment{Problem}{\begin{problem}}{\end{problem}}
    \newenvironment{Corollary}{\begin{corollary}}{\end{corollary}}
    \newenvironment{Example}{\begin{example}}{\end{example}}
    \newenvironment{Examples}{\begin{examples}}{\end{examples}}
    \newenvironment{Definition}{\begin{definition}}{\end{definition}}
\makeatother











% ============================================================
% Title
% ============================================================

\title[]{Modèles aléatoires discrets}
\subtitle{Chapitre I: Introduction}
\author{Pierre-Olivier Goffard}
\institute{
	   Université de Lyon 1\\
	ISFA\\
	   \texttt{pierre-olivier.goffard@univ-lyon1.fr}
	  }
\date{
ISFA\\
\today}
\lstset{language=SAS}
\begin{document}

\frame{\titlepage}


% ============================================================
\begin{frame}[allowframebreaks]
\underline{I. Introduction et motivations}\\
\begin{block}{Objectif}
Introduire la notion de processus stochastique.
\begin{itemize}
\item Chaine de Markov
\item Processus de Poisson
\item Processus de Poisson composé
\item Processus de branchement
\item Martingale à temps discret
\end{itemize}
\end{block}
\begin{definition}[Processus stochastique]
Un processus stochastique $\{X_t\text{ ; }t\}$ est une suite de variables aléatoires indicée sur le temps.
\begin{itemize}
\item Si $t\in\mathbb{N}$ alors on parle de processus stochastique en temps discret.
\item Si $t\in\mathbb{R}_+$ alors on parle de processus stochastique en temps continu.
\end{itemize}
$\{X_t\text{ ; }t\}$ est une quantité qui évolue de façon aléatoire dans le temps.
\end{definition}
% \begin{example}
% \begin{itemize}
% \item La réserve financière d'une compagnie d'assurance
% \item Le cours d'une action
% \item Le nombre de client à un guichet
% \item Le temps qu'il fait dehors
% \item Le nombre d'infectés au cours d'une épidémie
% \end{itemize}
% \end{example}
\begin{example}[Modèle de ruine à temps discret]
Soit une compagnie d'assurance non-vie,
\begin{itemize}
\item detenant un capital initial $u>0$,
\item récupérant $c>0$ sur chaque période d'exercice au titre des primes,
\item indemnisant $X$, variable aléatoire positive, à ses assurés sur chaque période d'exercice.
\end{itemize}
Soit $\{R_t\text{ ; }t\in\mathbb{N}\}$ la valeur de la réserve financière à la fin de chaque période d'exercice. On a
$$
R_0=u\text{ et }R_t=u+c\times t - \sum_{k=1}^{t}X_k,\text{ }t=1,2,3,\ldots
$$
où $X_1,X_2,\ldots$ sont des variables aléatoires i.i.d. positives distribuées comme $X$. On s'intéresse à la probabilité de ruine avant $T$ définit par
$$
\psi(u,T)=\mathbb{P}(R_t<0,\text{ pour un certain }t\leq T)
$$
\begin{itemize}
\item Calibrer $c$ $\Rightarrow$ tarification, typiquement
$$
c=\mathbb{E}(X)(1+\eta),
$$
avec un chargement de sécurité $\eta>0$.
\item Calibrer $u$ $\Rightarrow$ provisionnement. Choisir $u$ grand permet d'éviter une ruine à court terme.
\end{itemize}
\end{example}
\end{frame}
\begin{frame}[allowframebreaks]
\begin{example}[Le cours de l'action Amazon]
Soit $\{X_t\text{ ; }t\in\mathbb{N}\}$ la valeur journalière de l'action Amazon à la clotûre du marché. On suppose que le cours de l'action
\begin{itemize}
\item augmente de $\alpha\%$ avec probabilité $p$
\item diminue de $\beta\%$ avec probabilité $1-p$
\end{itemize}
Soit $N$ le nombre de jour d'augmentation, à l'instant $t$, l'action vaut
$$
X_t=(1+\alpha)^{N}(1-\beta)^{t-N}X_0
$$
Quelle est la loi de $N$ en fonction de $t$ et $p$?\\
On peut raffiner le modèle en supposant q'une augmentation est plus probable si une augmentation est observée le jour d'avant. \\
On définit le processus $\{Y_t\text{ ; }t\in\mathbb{N}\}$ indiquant si le cours de l'action augmente ou diminue.
\begin{itemize}
\item L'espace d'état (valeur possibles de $Y_t$) est $E=\{\text{up, down}\}$
\item On a, pour $t\in\mathbb{N}$,
$$
\left(
\begin{array}{cc}
\mathbb{P}(Y_{t+1}=\text{ up}|Y_{t}=\text{ up})=p& \mathbb{P}(Y_{t+1}=\text{ down}|Y_{t}=\text{ up})=1-p\\
\mathbb{P}(Y_{t+1}=\text{ up}|Y_{t}=\text{ down})=1-q& \mathbb{P}(Y_{t+1}=\text{ down}|Y_{t}=\text{ down})=q
\end{array}
\right)
$$
\end{itemize}
Lorsque l'état du processus ne dépend que de la valeur précédente, on parle de processus de Markov. $\{Y_t\text{ ; }t\in\mathbb{N}\}$ définit une chaine de Markov. On parle de chaine de Markov cachée influençant la valeur du processus $X_t$. Déssiner le graphe de transition.
\end{example}
\begin{example}[Le modèle SIR]
Soit une population de $N$ individus placés dans $3$ compartiments
\begin{itemize}
\item Compartiment \textbf{S} pour \textit{Susceptibles}
\item Compartiment \textbf{I} pour \textit{Infected}
\item Compartiment \textbf{R} pour \textit{Removed}
\end{itemize}
On modélise le nombre d'individus dans chaque compartiment au cours du temps via les processus $S_t$, $I_t$ et $R_t=N-(S_t+I_t)$. Le modèle évolue selon deux règles, durant chaque période,
\begin{itemize}
\item Chaque infecté contamine un susceptible donné avec une probabilité $p$
\item Les infectés vont dans le compartiment \textbf{R}
\end{itemize}
Chaque susceptible est contaminé durant une période donnée avec une probabilité $1-(1-p)^{I_t}$, le nombre de susceptible $S_{t+1}$ à $t+1$ dépend du nombre de susceptibes $S_t$ et du nombre d'infecté $I_t$ à l'instant $t$ avec
$$
S_{t+1}\sim\text{Bin}(S_t, (1-p)^{I_t})
$$
L'épidémie s'arrête à $T$ lorsque le nombre d'infectés est $0\text{ }(I_T=0)$, on s'intéresse à la taille finale de l'épidémie donnée par $N-S_T$, soit combien de susceptible ont été contaminé. On a une fois de plus un processus $(S_t,I_t)$ avec une dépendance à l'état précédent, donc Markovien.
\end{example}
\end{frame}


\end{document}
